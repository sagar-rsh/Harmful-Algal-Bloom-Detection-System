\documentclass[conference]{IEEEtran}
\IEEEoverridecommandlockouts
% The preceding line is only needed to identify funding in the first footnote. If that is unneeded, please comment it out.
%Template version as of 6/27/2024

\usepackage{cite}
\usepackage{amsmath,amssymb,amsfonts}
\usepackage{algorithmic}
\usepackage{graphicx}
\usepackage{textcomp}
\usepackage{xcolor}
\usepackage{booktabs}
\usepackage{longtable}
\usepackage{hyperref}
\usepackage{float}

\def\BibTeX{{\rm B\kern-.05em{\sc i\kern-.025em b}\kern-.08em
    T\kern-.1667em\lower.7ex\hbox{E}\kern-.125emX}}

\begin{document}

\title{HAB (Harmful Algal Bloom) Detection System\\
{\footnotesize COMP47250 Team Software Project Final Report}
}

\author{\IEEEauthorblockN{Daniel Ilyin}
\IEEEauthorblockA{\textit{School of Computer Science, UCD} \\
Dublin, Ireland \\
daniel.ilyin@ucdconnect.ie}
\and
\IEEEauthorblockN{Sagar Satish Poojary}
\IEEEauthorblockA{\textit{School of Computer Science, UCD} \\
Dublin, Ireland \\
email address @ucdconnect.ie}
\and
\IEEEauthorblockN{Dharmik Arvind Vara}
\IEEEauthorblockA{\textit{School of Computer Science, UCD} \\
Dublin, Ireland \\
email address @ucdconnect.ie}
\and
\IEEEauthorblockN{Karthika Garikapati}
\IEEEauthorblockA{\textit{School of Computer Science, UCD} \\
Dublin, Ireland \\
email address @ucdconnect.ie}
\and
\IEEEauthorblockN{Kruthi Jangam}
\IEEEauthorblockA{\textit{School of Computer Science, UCD} \\
Dublin, Ireland \\
email address @ucdconnect.ie}
\and
\IEEEauthorblockN{Roshan Palem}
\IEEEauthorblockA{\textit{School of Computer Science, UCD} \\
Dublin, Ireland \\
email address @ucdconnect.ie}
}

\maketitle

\begin{abstract}

\end{abstract}

\begin{IEEEkeywords}

\end{IEEEkeywords}

\section{Introduction}

\section{User Stories and Scenarios}

\subsection{Marine Biologists and Environmental Agencies}

\subsection{Aquaculture Operators}

\subsection{Public Health Officials}

\section{State of the Art and Related Work}

\subsection{Traditional HAB Detection Methods}

\subsection{Remote Sensing Approaches}

\subsection{Machine Learning in Environmental Monitoring}

\subsection{Spatiotemporal Analysis Methods}

\subsection{Comparison with Existing Systems}

\section{Implementation}

\subsection{System Architecture Overview}

\subsection{Data Collection and Preprocessing Pipeline}

\subsubsection{Ground Truth Data Sources}

\subsubsection{Remote Sensing Data Acquisition}

\subsubsection{Data Preprocessing and Quality Control}

\subsection{Datacube Construction}

\subsubsection{Spatiotemporal Window Definition}

\subsubsection{Modality Selection and Integration}

\subsubsection{Sparse Data Handling}

\subsection{Machine Learning Architecture}

\subsubsection{CNN Feature Extraction}

\subsubsection{Temporal Classification Methods}

\subsubsection{Model Training and Validation}

\subsection{Web-Based Dashboard}

\subsubsection{Frontend Implementation}

\subsubsection{Real-time Data Integration}

\subsubsection{Visualization Components}

\subsection{Cloud Infrastructure and Deployment}

\subsubsection{AWS Architecture}

\subsubsection{MLOps Pipeline}

\subsubsection{Scalability and Performance Optimization}

\section{Evaluation and Results}

\subsection{Dataset Description}

\subsection{Experimental Setup}

\subsubsection{Cross-Validation Strategy}

\subsubsection{Performance Metrics}

\subsubsection{Baseline Comparisons}

\subsection{Detection Performance}

\subsubsection{Classification Results}

\subsubsection{Feature Importance Analysis}

\subsubsection{Comparison with Traditional Methods}

\subsection{Prediction Performance}

\subsubsection{Temporal Forecasting Results}

\subsubsection{Prediction Horizon Analysis}

\subsection{System Performance Evaluation}

\subsubsection{Scalability Testing}

\subsubsection{Response Time Analysis}

\subsubsection{Resource Utilization}

\subsection{User Evaluation}

\subsubsection{User Study Design}

\subsubsection{Usability Assessment}

\subsubsection{User Feedback and Insights}

\subsection{Technical Evaluation}

\subsubsection{Model Performance Analysis}

\subsubsection{Infrastructure Performance}

\subsubsection{System Reliability}

\section{Conclusions and Future Work}

\subsection{Summary of Achievements}

\subsection{Key Technical Contributions}

\subsection{Limitations and Challenges}

\subsection{Future Directions}

\subsubsection{Model Improvements}

\subsubsection{Data Enhancement}

\subsubsection{System Extensions}

\subsection{Impact and Applications}

\section*{Acknowledgment}

The preferred spelling of the word ``acknowledgment'' in America is without 
an ``e'' after the ``g''. Avoid the stilted expression ``one of us (R. B. 
G.) thanks $\ldots$''. Instead, try ``R. B. G. thanks$\ldots$''. Put sponsor 
acknowledgments in the unnumbered footnote on the first page.

\begin{thebibliography}{00}
\bibitem{b1} G. Eason, B. Noble, and I. N. Sneddon, ``On certain integrals of Lipschitz-Hankel type involving products of Bessel functions,'' Phil. Trans. Roy. Soc. London, vol. A247, pp. 529--551, April 1955.
\bibitem{b2} J. Clerk Maxwell, A Treatise on Electricity and Magnetism, 3rd ed., vol. 2. Oxford: Clarendon, 1892, pp.68--73.
\end{thebibliography}

\end{document}